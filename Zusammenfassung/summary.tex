\documentclass{report}
\usepackage{graphicx}
\graphicspath{ {./screenshots/} }
\usepackage{titlesec}
\usepackage{float}
\usepackage{amsmath}
\usepackage{amssymb}
\usepackage{listings}
\usepackage{pifont}

\titleformat{\chapter}[display]
  {\normalfont\bfseries}{}{0pt}{\Large}
\titlespacing*{\chapter}{0pt}{-100pt}{25pt}


\begin{document}

\begin{center}
    \large Zusammenfassung
\end{center}

\begin{center}
    \Large \textit{Maschinelles Lernen}\\
    \vspace*{.5em}
    \normalsize \textit{WS 19/20}\\
    \vspace*{45em}
    \large \today
\end{center}

\newpage

\chapter{Grundlagen}
\vspace*{-1.25em}
\section{Lineare Algebra}
\subsection{Skalarprodukt}
- Vektoren $x, y \in \mathbb{R}^n$: $x\circ y = \sum_{i=1}^n x_i\cdot y_i = x^Ty$\\
- $\begin{bmatrix}1\\2\end{bmatrix}\circ \begin{bmatrix}3\\4\end{bmatrix} = 1\cdot 3 + 2\cdot 4 = 11$
\subsection{Vektornorm}
$f: \mathbb{R}^n\rightarrow \mathbb{R}$ mit\\
\vspace*{-1.5em}
\begin{itemize}
  \item $f(x) = 0 \Rightarrow x = 0$
  \item $f(x + y) \leq f(x) + f(y)$ (\textit{Dreiecksgleichung})
  \item $f(\alpha x) = |\alpha|f(x)$
\end{itemize}
- $L_1$-Norm: $||x||_1 = \sum_i|x_i|$\\
- $L_2$-Norm: $||x||_2 = \sqrt{\sum_i x_i^2}$ (\textit{euklidische Norm})
\subsection{Matrizen}
- \textit{m} Zeilen und \textit{n Spalten}
A = $\begin{bmatrix}A_{11} & ... & A_{1n}\\A_{m1} & ... & A_{mn} \end{bmatrix}$,
$\begin{bmatrix}1 & 2\\3 & 4\\5 & 6\end{bmatrix}^T = \begin{bmatrix}1 & 3 & 5\\2 & 4 & 6\end{bmatrix}$\\
- $\begin{bmatrix}a & b & c\\d & e & f\end{bmatrix}\cdot \begin{bmatrix}g & h\\i & j\\k & l\end{bmatrix}$
= $\begin{bmatrix}ag + bi + ck & ah + bj + cl\\dg + ei + fk & dh + ej + fl\end{bmatrix}$,
$I = \begin{bmatrix}1 & 0 & 0\\0 & 1 & 0\\0 & 0 &1\end{bmatrix}$\\
- $A^{-1}A = I$ (Matrizen mit linear abhängigen Zeilen oder Spalten (niedriger Rang) sind nicht invertierbar)
\subsection{Hyperebene}
- x $\in$ $\mathbb{R}^d$ erfüllen Gleichung $w_0 + w_1x_1 + w2_x2 + ... + w_dx_d = 0$ ($w_0 + w^Tx = 0$)\\
- $d = 1$: Skalar ($w_0 + w_1x_1$), $d = 2$: Gerade ($w_0 + w_1x_1 + w_2x_2$), $d = 3$: Ebene\\
- Für einen Punkt $x$ entscheidet das Vorzeichen $sgn(w_0 + w^Tx)\in \{-1, 0, 1\}$ auf welcher Seite der Hyperebene er liegt (bzw. ob er auf ihr liegt)\\
\begin{center}
\includegraphics[scale=.2]{ml01_1} $1 - \frac{1}{4}x_1 - \frac{1}{6}x_2 = 0$
\end{center}

\section{Statistik}
\begin{itemize}
  \item Durchschnittswert: (Summe über alle Zeilen) / (Anzahl an Zeilen)
  \item Standardabweichung: Wurzel ver Varianz
  \item 25\%-Quantile: 25\% aller Werte sind kleiner als dieser Wert
  \item 50\%-Quantile: 50\% aller Werte sind kleiner als dieser Wert (= \textit{Median})
  \item 75\%-Quantile: 75\% aller Werte sind kleiner als dieser Wert
\end{itemize}


\section{Analysis}
\subsection{Kettenregel}
- Wenn $z$ von $y$ und $y$ von $x$ abhängt, dann gilt: $\frac{dz}{dx} = \frac{dz}{dy}\frac{dy}{dx}$\\
- $f(x) = g(h(x)) = \frac{1}{2}\cdot(x_1 - x_2)^2$ $\rightarrow$ $g(x) = \frac{1}{2}x^2$ und $h(x) = x_1 - x_2$\\
- $\frac{df}{dx_2} = \frac{dg}{dh}\frac{dh}{d_x2} = h(x)(-1) = -(x_1 - x_2) = x_2 - x_1$
\subsection{Partielle Ableitung}
$f(x) = 2x_1^3 - 5x_2^2 + 3$, $\frac{df}{dx_1} = 6x_1^2$, $\frac{df}{dx_2} = -10x_2$
\subsection{Gradient}
$\nabla f = \begin{bmatrix}\frac{df}{dx_1}\\ . \\. \\.\\\frac{df}{dx_n}\end{bmatrix}$, $f(x) = 2x_1^3 - 5x_2^2 + 3, \nabla f = \begin{bmatrix}6x_1^2\\-10x_2\end{bmatrix}$

\section{Was ist maschinelles Lernen}
\subsection{Paradigmenwechsel}
Es ist schwierig, den entsprechenden Programmcode manuell zu schreiben, daher wird ein anderes Paradigma verwendet:
\begin{center}
  \includegraphics[scale=.4]{ml01_2}
\end{center}
Drei verschiedene Lernmethoden
\begin{itemize}
  \item Überwachtes Lernen (\textit{Supervised Learning})
  \item Unüberwachtes Lernen (\textit{Unsupervised Learning})
  \item Bestärkendes Lernen (\textit{Reinforcement Learning})
\end{itemize}

\section{Überwachtes Lernen}
- Ziel: finden einer Funktion $f: X \rightarrow Y$ wobei $X$ auch \textit{Features / Prädiktoren} und $Y$ auch \textit{Responses} genannt werden\\
- $X$ = $\mathbb{R}^d$ (\textit{d}-dimensionaler Vektorraum) mit $d\in \mathbb{N}$\\
- Eine perfekte Abbildung ist nicht möglich, es treten \textit{reduzierbare} Fehler (z.B. durch eine bessere Funktion $f$) und \textit{nicht reduzierbare} Fehler (z.B. Messfehler in Eingabedaten) auf\\
\vspace*{-1.5em}
\begin{itemize}
  \item \textit{Vorhersage:} $y = f(x)$ optimieren wobei $f$ auch \textit{Blackbox} sein kann
  \item \textit{Inferenz:} Interpretierbarkeit von $f$ steht im Vordergrund (Welche Prädiktoren sind für welche Response verwantwortlich)
  \item \textit{Parametrische} Methoden: Annahme einer parametrisierten Struktur von $f$ dessen Parameter mit Hilfe von Daten bestimmt werden
  \item \textit{Nicht-parametrische} Methoden: Keine Annahme einer Struktur von $f$ sondern möglichst direkte Definition mit Hilfe von Daten
\end{itemize}
- Menge $X$ und $Y$ bekannt, genaue Abbildung $f$ kann aber nur anhand von Beispielen $D = \{(x^i, y^i) | x^i \in X, y^i \in Y, 1 \leq i \leq n\}$ (\textit{Trainingsdatensatz} bzw. \textit{gelabelte} Daten) erahnt werden\\

\subsection{Beispiel Klassifikation}
- Wenn $Y$ diskrete Menge $\{C_1, ..., C_k\}$ für $k \in \mathbb{N}$ dann handelt es sich um ein \textit{Klassifikationsproblem}, $C_1, ..., C_k$ sind dann \textit{Klassen / Kategorien}\\
- $|Y| = 2$ (\textit{Binäre} Klassifikation) mit $f : \mathbb{R} \rightarrow \{$angenehm, unangenehm$\}$ (Temperaturklassifikation)\\
- $|Y| = 5$ (\textit{Mehrklassen}-Klassifikation) mit $f : \mathbb{R} \rightarrow \{$frostig, kalt, angenehm, warm, heiß$\}$

\subsection{Beispiel Regression}
- Wenn $Y$ kontinuierliche Menge, d.h. $Y \subseteq \mathbb{R}$, dann handelt es sich um ein \textit{Regressionsproblem}\\
- Interesse an \textit{quantitativen} Aussagen\\
\begin{center}
  \includegraphics[scale=.4]{ml01_3}
\end{center}
- Ausgabemenge $Y$ kann auch mehrdimensional sein (z.B. $\{$gut, schlecht$\}\times \{$günstig, normal, teuer$\}$)

\section{Unüberwachtes Lernen}
- Mehrwert erhalten ohne Zuhilfenahme von gelabelten Daten\\
- Man geht von Menge an Daten $D = \{x^i|x^i \in X, 1 \leq i \leq n\}$ aus und versucht mehr über Beschaffenheit von $X$ herauszufinden\\
- z.B. \textit{Verteilung} von $X$ bei Sprachmodellen, \textit{Dimensionsreduktion} zur Verbesserung von überwachten Lernverfahren

\section{Datenvisualisierung}
\begin{figure}[H]
  \centering
  \begin{minipage}[b]{0.4\textwidth}
    \includegraphics[scale=.25]{ml01_4}
  \end{minipage}
  \hfill
  \begin{minipage}[b]{0.4\textwidth}
    \includegraphics[scale=.25]{ml01_5}
  \end{minipage}
\end{figure}
\begin{figure}[H]
  \centering
  \begin{minipage}[b]{0.4\textwidth}
    \includegraphics[scale=.25]{ml01_6}
  \end{minipage}
  \hfill
  \begin{minipage}[b]{0.4\textwidth}
    \includegraphics[scale=.25]{ml01_7}
  \end{minipage}
\end{figure}
\subsection{Boxplot}
\begin{center}
  \includegraphics[scale=.125]{ml01_8}
\end{center}
- Zwischen dem linken waagerechten Strich (Minimum) und dem rechten waagerechten Strich (Maximum) liegen 99.3\% aller Daten\\
- Die \textit{Outliers} an den beiden Enden sind die letzten 0.7\%\\
- Der Abstandsfaktor (hier 1.5) ist frei wählbar\\
- Sollte der Punkt $Q1 - 1.5\cdot IQR$ bzw. $Q3 + 1.5\cdot IQR$ nicht existieren wird der Strich auf den nächst-näheren Punkt gesetzt

\section{Datenvorverarbeitung}
Bevor ein Modell erstellt und trainiert werden kann, müssen Daten durch\\
\vspace*{-1.5em}
\begin{itemize}
  \item \textit{Auswahl}: Nur für den Anwendungsfall relevante Daten verwenden
  \item \textit{Aufbereitung}
  \subitem Dateiformat (Tabellen, BigData)
  \subitem Bereinigung von unvollständigen oder ungültigen Daten
  \subitem Repräsentative Auswahl bei langer Laufzeit / großem Speicheraufwand
  \item \textit{Transformation}
  \subitem Features in geeigneten Wertebereich bringen ([0, 1])
  \subitem Zerlegen in sinnvolle Features
  \subitem Aggregation mehrerer Features
\end{itemize}

\chapter{Lineare Regression}
\section{Lineare Regression im Eindimensionalen}
- $f : \mathbb{R} \rightarrow \mathbb{R}$ mit $f_w(x) = w_1x + w_0$\\
- $w = (w_0, w_1)^T \in \mathbb{R}^2$ sind die \textit{Parameter} des Modells
\begin{center}
  \includegraphics[scale=0.275]{ml02_1}
\end{center}
- Wie mit Daten $D = \{(x^i, y^i) \in \mathbb{R}^2 | 1 \leq i \leq n\}$ die \textit{besten} Parameter von $f$ bestimmen?\\
\subsection{Lösungsverfahren}
- Quadratischen Fehler (\textit{Residual Sum of Squares}) mit\\
$RSS(w) = \sum_{i=1}^n(y^i - f_w(x^i))^2$ bestimmen
\begin{center}
  \includegraphics[scale=0.35]{ml02_2}
\end{center}
- Zur besseren Vergleichbarkeit verwendet man oft die normalisierte Variante\\
\textit{Mean Squared Error}: $MSE(w) = \frac{1}{n}\cdot RSS(w)$ (n = Anzahl Trainingsdaten)\\
- Die beste Funktion durch Minimierung des Fehlers finden
$\Rightarrow w^* =$ arg min $E(w) =$ arg min $\frac{1}{2}\cdot \sum_{i=1}^n(y^i - f_w(x^i))^2$\\
\vspace*{-1.25em}
\begin{itemize}
  \item Ableitung von $E(w)$ gleich Null setzen und Gleichungssystem lösen
  \item $\nabla E(w) = \begin{bmatrix}\frac{dE(w)}{dw_0}\\\frac{dE(w)}{dw_1}\end{bmatrix} = 0$
  \subitem $\frac{dE(w)}{dw_0} = -\sum_{i=1}^ny^i + w_1\cdot \sum_{i=1}^nx^i + n\cdot w_0$
  \subitem $\frac{dE(w)}{dw_1} = -\sum_{i=1}^nx^iy^i + w_1\cdot \sum_{i=1}^nx^ix^i + w_0\cdot \sum_{i=1}^nx^i$
  \item Gleichungssystem mit zwei Gleichungen und zwei Unbekannten lösbar, aber numerisch ungenau bei großen Matrizen 
\end{itemize}

\subsection{Gradientenabstiegsverfahren}

\begin{center}
  \includegraphics[scale=.25]{ml02_3}
\end{center}
- Iterativ einem Bruchteil der negativen Ableitung: $-\eta f'(x) = \eta\cdot (2 - 2x)$ folgen\\
- \textit{Lernrate} $\eta$ hat direkten Einfluss auf Konvergenz (zu klein $\Rightarrow$ viele Schritte, zu groß $\Rightarrow$ Oszillation)\\
\begin{lstlisting}
  w0 = 0, w1 = 0
  for (x, y) in D
    dw0 += -y + w1*x + w0
    dw1 += -xy + w1*x*x + w0*x
  end for
  w0 += -eta*dw0
  w1 += -eta*dw1
\end{lstlisting}

\section{Mehrdimensionale Lineare Regression}
- $X = \mathbb{R}^d$ und $f: \mathbb{R}^d \rightarrow \mathbb{R}$ sowie $f_w(x) = \sum_{i=1}^dw_ix_i + w_0$\\
mit Parametern $w = (w_0, w_1, ..., w_d)^T \in \mathbb{R}^{d + 1}$
- Kompaktere Schreibweise mit $x_0 = 1$: $f_w(x) = \sum_{i = 1}^dw_ix_i + w_0 = w^Tx$\\
\begin{center}
  \includegraphics[scale=.25]{ml02_4}
\end{center}
- Im Mehrdimensionalen wird eine Hyperebene, im dreidimensionalen eine Ebene, im Raum so positioniert, dass der Abstand zu den Datenpunkten minimiert wird\\
- Angepasste Fehlermetrik $E(w) = \frac{1}{2}\cdot \sum_{i=1}^n(y^i - f(x^i))^2$
mit $\nabla E(w) = \begin{bmatrix}\frac{dE(w)}{dw_0}\\\frac{dE(w)}{dw_1}\\...\\\frac{dE(w)}{dw_d}\end{bmatrix}$\\
\begin{lstlisting}
  dw = 0
  for (x, y) in D
    dw += -(y - f(x) * gradF(x))
  end for
  w += -eta*dw
\end{lstlisting}
wobei gradF(x) = $\nabla f(x) = \begin{bmatrix}1\\x_1\\...\\x_d\end{bmatrix}$\\
\begin{center}
  \includegraphics[scale=.25]{ml02_5}
\end{center}

\section{Genauigkeit}
- Wie gut ist das durch das Gradientenabstiegsverfahren gefundene Modell?\\
$\Rightarrow$ Quadratischer Fehler \textit{RSS} oder mittlerer quadratischer Fehler \textit{MSE}\\
- Letzterer ist unabhängig von der Anzahl an Trainingsdaten allerdings gibt es keine allgemein gültige Skala da diese vom Wertebereich der y-Werte abhängt

\subsection{$R^2$ Statistik}
- Definiert über den quadratischen Gesamtfehler $TSS = \sum_{i=1}^n(y^i - \bar{y})^2$\\
- $\bar{y} = \frac{1}{n}\cdot \sum_{i=1}^ny^i$ $\Rightarrow$ $R^2(w) = \frac{TSS - RSS(w)}{TSS} = 1 - \frac{RSS(w)}{TSS}$\\
- \textit{TSS} misst die komplette Varianz in den Ausgabedaten $y^i$\\
- $TSS - RSS(w)$ misst die durch das Modell mit Parametern $w$ erklärte Varianz\\
- $R^2$ misst die komplette Varianz des Modells und ist $\in [0, 1]$\\
\vspace*{-1.25em}
\begin{itemize}
  \item $R^2$ nahe 1 zeugt von einem passenden Model das die Daten gut erklärt (viele Datenpunkte liegen auf der Geraden bzw. Hyperebene)
  \item $R^2$ nahe 0 bedeutet, dass das Modell die Daten schlecht erklärt (umso weiter entfernt die Datenpunkte von der Hyperebene sind umso näher ist $R^2$ bei 0)
\end{itemize}
- $R^2$ ist unabhängig von Anzahl an Trainingsdaten \textit{UND} dem Wertebereich\\
- Allgemeine Aussage ab welchem $R^2$-Wert das Modell \textit{gut} ist, ist nicht möglich. Hängt vom Anwendungsfall (Medizin / Physik) ab

\section{Interpretierbarkeit}
- Die Parameter $w$ von Linearen Regressionsmodellen sind interpretierbar:\\
\vspace*{-1.25em}
\begin{itemize}
  \item $w_i > 0$: positiver Zusammenhang, steigt $x_i$ um $m$ so steigt $y$ um $m\cdot |w_i|$
  \item $w_i nahe 0$: kein linearer Zusammenhang zwischen $x_i$ und $y$
  \item $w_i < 0$ negativer Zusammenhang, steigt $x_i$ um $m$ so sinkt $y$ um $m\cdot |w_i|$
\end{itemize}

\section{Nichtlineare Zusammenhänge}
- Mit der mehrdimensionalen linearen Regressions lassen sich auch \textit{nichtlineare} Zusammehänge lernen\\
- Mit Funktion $\Phi: \mathbb{R} \rightarrow \mathbb{R}^d$ wird ein \textit{Basiswechsel} vollzogen\\
\vspace*{-1.25em}
\begin{itemize}
  \item Die Konkatenation von $\Phi: \mathbb{R} \rightarrow \mathbb{R}^d$, $\Phi(x) = (x, x^2)^T$
  und $f: \mathbb{R}^2 \rightarrow \mathbb{R}, f(x) = w_2x_w + w_1x_1 + w_0$ durch $f\circ \Phi$
  erlaubt Darstellung der quadratischen Funktion $(f\circ \Phi)(x) = f(\Phi(x)) = w_2x^2 + w_1x + w_0$
  \item $\Phi: \mathbb{R}^2 \rightarrow \mathbb{R}^5, \Phi(x) = (x_2, x_1, x_1x_2, x_2^2, x_1^2)^T$\\
  und $f: \mathbb{R}^5 \rightarrow \mathbb{R}, f(x) = \sum_{i = 1}^5 w_ix_i + w_0$ ergibt $(f\circ \Phi)(x) = f(\Phi(x)) = w_5x_1^2 + w_4x_2^2 + w_3x_1x_2 + w_2x_1 + w_1x_2 + w_0$
\end{itemize}

\subsection{Beispiel}
- Annahme eines quadratischen Zusammenhangs $f(x) = w_2\cdot x^2 + w_1\cdot x + w_0$
\begin{center}
  \includegraphics[scale=.295]{ml02_6}
\end{center}

\subsection{Richtiger Grad}
- Mit der mehrdimensionalen Regression, dem Basiswechsel und Gradientenabstiegsverfahren ist es möglich, ein Polyon \textit{n}-ten Grades an \textit{n} Datenpunkte zu fitten
\begin{figure}[H]
  \centering
  \begin{minipage}[b]{0.4\textwidth}
    \includegraphics[scale=.2125]{ml02_7}
  \end{minipage}
  \hfill
  \begin{minipage}[b]{0.4\textwidth}
    \includegraphics[scale=.2125]{ml02_8}
  \end{minipage}
\end{figure}

- Mit höherer Modellkomplexität (Grad und Koeffizienten des Polynoms) kommt es zu\\
\vspace*{-1.25em}
\begin{itemize}
  \item Numerischen Problemen
  \item \textit{Overfitting}: Das Modell passt sich zu sehr an die Daten an und ist nicht mehr in der Lage zu generalisieren
  $\rightarrow$ Schlechte Leistung in der Praxis
\end{itemize}

\section{Trainings- und Testdaten}
- Datensatz $D$ wird in zwei disjunkte Teile $T$ und $V$ aufgeteilt\\
- Trainingsdatensatz $T$ wird für das Lernen verwendet\\
- Testdatensatz $V$ enthält ungesehene Daten zur Validierung der Praxistauglichkeit\
\vspace*{-1.25em}
\begin{itemize}
  \item Ein hoher Fehler auf $T$ lässt auf Unteranpassung schließen (zu geringe Modellkomplexität, zu wenig Daten)
  \item Ein geringer Fehler auf $T$ aber hoher Fehler auf $V$ bedeutet Überanpassung $\rightarrow$ Komplexität verringern
\end{itemize}

\section{Optimierung von Hyperparametern}
- Lineare Regression auf Polynomen mit Gradientenabstiegsverfahren besitzt\\
\vspace*{-1.25em}
\begin{itemize}
  \item Lernrate $\eta$: Einfluss auf Modellkomplexität
  \item Anzahl Lernschritte: Je geringer desto unwahrscheinlicher ist Überanpassung, allerdings Unteranpassung wiederum möglich
  \item Polynomgrad Zu Hoch $\rightarrow$ Überanpassung, zu niedrig $\rightarrow$ Unteranpassung
\end{itemize}
als Hpyerparameter

\subsection{Rastersuche}
- Durchsuchen des Hyperparameterraums entweder\\
\vspace*{-1.25em}
\begin{itemize}
  \item Entlang eines gleichmäßigen Rasters mit linearer oder logarithmischer Skala
  \item Entlang eines zufälligen Rasters mit uniformer oder logarithmischer Skala
\end{itemize}
- Verfeinern der Suche durch Rekursion

\subsection{Validierungsdaten}
- Sollen die Hyperparameter des Modells optimiert werden, werden die verfügbaren Daten $D$
in \textit{Trainingsdaten}, \textit{Validierungsdaten} und \textit{Testdaten} aufgeteilt.\\
- Die Hyperparameter werden mit dem Validierungsdatensatz optimiert
- Endgültige Performance des Modells wird auf den Testdaten bestimmt

\subsection{Kreuzvalidierung}
- Zerteilen des Datensatzes in $k$ Partitionen, ws wird nun $k$-mal trainiert\\
- Mit jeder Iteration $i$ wird eine andere Partition $i$ getestet\\
- Die Restlichen Partitionen dienen als Trainingsdaten\\
- Final wird die ausgewählte Leistungsmetrik über $k$ Iterationen gemittelt
\begin{center}
  \includegraphics[scale=.225]{ml02_9}
\end{center}

\subsection{Ridge Regression}
- Verhindern von Überanpassung durch Bestrafung von $w$ für exzessive Werte mit angepasster Fehlerfunktion
$E(w) = \frac{1}{2}\cdot \sum_{i=1}^n(y^i - f_w(x^i))^2 + \alpha ||w||^2$\\
- Hyperparamter $\alpha \in \mathbb{R} \geq 0$ ist ein weiterer Freiheitsgrad mit dem sich der Polynomgrad stufenlos einstellen lässt
\begin{figure}[H]
  \centering
  \begin{minipage}[b]{0.4\textwidth}
    \includegraphics[scale=.215]{ml02_10}
  \end{minipage}
  \hfill
  \begin{minipage}[b]{0.4\textwidth}
    \includegraphics[scale=.215]{ml02_11}
  \end{minipage}
\end{figure}
\begin{itemize}
  \item $\alpha = 0$: klassische Regression
  \item $\alpha > 0$: Normaler Wirkungsbereich, mit wachsendem $\alpha$ werden $w$ immer weiter eingeschränkt und der effektive Polynomgrad sinkt
  \item lim $\alpha \rightarrow \infty$: $f(x) = 0$ da Parameter lim $w \rightarrow 0$
\end{itemize}

\chapter{Logistische Regression}
\section{Klassifikation}
\subsection{Lineare Regression}
- Lineare Regression als Klassifikator zu verwenden: $f: \mathbb{R} \rightarrow \mathbb{R}$
mit $f(x) = w_1x + w_0$
\begin{center}
  \includegraphics[scale=.215]{ml03_1}
\end{center}
- Im Beispiel: $x$ = Monatseinkommen, $f(x)$ = Kunde kreditwürdig ja / nein, ABER:\\
\vspace*{-1.25em}
\begin{itemize}
  \item Diskrete Ausgabewerte (Klasse 0 / 1) wird nicht eingehalten, Ausgabe nimmt alle Werte in $[-0.0014, 1.4]$ an
  \item Intepretation von $f(x)$ als Wahrscheinlichkeit auch nicht möglich\\
  da $f(0) = -0.0014 < 0$ und $f(8000) = 1.4 > 1$
\end{itemize}

\subsection{Logistische Regression}
- Idee: Wahrscheinlichkeit der Klassenzugehörigkeit aufgreifen aber\\
Wertebereich von $f$ mit Hilfe der logistischen Funktion
$$logistic(x) = \frac{e^x}{1 + e^x}$$
unter Kontrolle bekommen
\begin{center}
  \includegraphics[scale=.3]{ml03_2}
\end{center}
- Kombination der Linearen Regression $f(x) = w_1x + w_0$ mit der logistischen Funktion
$$p(x) = logistic(f(x)) = \frac{e^{w_1x + w_0}}{1 + e^{w_1x + w_0}}$$
$$p(x) \in (0, 1) \forall x \in \mathbb{R}$$\\
- $p(x)$ ist die Wahrscheinlichkeit, dass $x$ zur Klasse $1$ gehört:\\
$$p(x) = Pr(y = 1 | X = x)$$\\
- $x$ gehört zur Klasse $0$ mit Wahrscheinlichkeit $1 - p(x)$:\\
$$Pr(y = 0|X = x) = 1 - Pr(y = 1 | X = x) = 1 - p(x)$$

\section{Maximum Likelihood}
Parameter der Modells werden so bestimmt, dass die Wahrscheinlichkeit, dass das Modell die Daten generiert, maximiert wird\\

\subsection{Beispiel Münzwurf}
- Eine Münze zeigt mit unbekannter Wahrscheinlichkeit $w \in [0, 1]$ Kopf und mit Wahrscheinlichkeit $(1 - w)$ Zahl\\
- Münze wird $n$-mal geworfen, die Wahrscheinlichkeit, $k$-mal Kopf zu erhalten ist\\
$$L(w) = w^k(1 - w)^{n-k}$$
- $L(w)$ wird \textit{Likelihood} genannt, man sucht: $arg$ $max$ $L(w)$\\
Maximum finden durch Ableiten\\
\vspace*{-1.25em}
\begin{itemize}
  \item $\frac{dL(w)}{dw} = kw^{k-1}\cdot(1 - w)^{n-k} + w^k(n-k)(1-w)^{n-k-1}\cdot(-1)$\\
  $ = w^{k-1}\cdot (1-w)^{n-k-1}\cdot[k(1-w)-(n-k)\cdot w]$
\end{itemize}
anschließend gleich Null setzen\\
\vspace*{-1.25em}
\begin{itemize}
  \item $\frac{dL(w)}{dw} = 0 \Leftrightarrow w = 0 \lor w = 1 \lor w = \frac{k}{n}$
  \item $k(1-w) - w\cdot(n - k) = 0 \Leftrightarrow k -wk - wn + wk = 0 \Leftrightarrow k - wn = 0 \Leftrightarrow w = \frac{k}{n}$
\end{itemize}
und Überprüfen von\\
\vspace*{-1.25em}
\begin{itemize}
  \item $w = 0: L(0) = 0^k(1 - 0)^{n-k} = 0$
  \item $w = 1: L(1) = 1^k(1 - 1)^{n-k} = 0$
  \item $w = \frac{k}{n}: L(\frac{k}{n}) = (\frac{k}{n})^k\cdot (1 - \frac{k}{n})^{n-k} > 0$ für $k > 0$ und $k \neq n$
\end{itemize}
Die \textit{Maximum Likelihood} Schätzung ist demnach $w = \frac{k}{n}$

\subsection{Beispiel Logistische Regression}
Die Likelihood ist wie folgt definiert
\begin{equation*}
L(w) = \Pi_{i=1}^n 
\begin{cases}
  p(x^i) & \text{falls $y^i = 1$}\\
  1 - p(x^i) & \text{falls $y^i = 0$}
\end{cases} = \Pi_{i | y^i=1}p(x^i) \cdot \Pi_{i | y^i = 0}(1 - p(x^i))
\end{equation*}
Anstatt das Maximum mit Hilfe der Ableitung zu finden, kann auch das Minimum des negativen Logarithmus gesucht werden
\begin{equation*}
  -log(L(w)) = -\sum_{i|y^i = 1}log(p(x^i)) - \sum_{i | y^i = 0}log(1 - p(x^i))
\end{equation*}
und Ableiten
$$\frac{d(-log(L_w))}{dw_j} = - \sum_{i|y^i = 1} \frac{\frac{dp(x^i)}{dw_j}}{p(x^i)} + \sum_{i|y^i = 0} \frac{\frac{dp(x^i)}{dw_j}}{1 - p(x^i)}$$
\begin{equation*}
  \frac{dp(x)}{dw_j} = \frac{d}{dw_j}\cdot \frac{e^{w_1x + w_0}}{1 + e^{w_1x + w_0}} = \frac{e^{w_1x + w_0}}{(1 + e^{w_1x + w_0})^2}
  \begin{cases}
    1 & \text{falls $j = 0$}\\
    $x$ & \text{falls $j = 1$}
  \end{cases}
\end{equation*}
$$1 - p(x) = \frac{1}{1 + e^{w_1x + w_0}}$$
\begin{equation*}
  \frac{\frac{dp(x)}{dw_j}}{p(x)} = \frac{1}{p(x)}\cdot \frac{dp(x)}{dw_j} = 
  \frac{1 + e^{w_1x + w_0}}{e^{w_1x + w_0}}\cdot \frac{e^{w_1x + w_0}}{(1 + e^{w_1x + w_0})^2} = (1 - p(x))
  \begin{cases}
    1 & \text{falls $j = 0$}\\
    $x$ & \text{falls $j = 1$}
  \end{cases}
\end{equation*}

\begin{equation*}
  \frac{\frac{dp(x)}{dw_j}}{1 - p(x)} = \frac{1}{1 - p(x)}\cdot \frac{dp(x)}{dw_j}
  = (1 + e^{w_1x + w_0})\cdot \frac{e^{w_1x + w_0}}{(1 + e^{w_1x + w_0})^2} = p(x)
  \begin{cases}
    1 & \text{falls $j = 0$}\\
    $x$ & \text{falls $j = 1$}
  \end{cases}
\end{equation*}

Zusammenfassend
$$\frac{d(-log(L(w)))}{dw_0} = -\sum_{i|y^i = 1}(1 - p(x^i)) + \sum_{i | y^i = 0}p(x^i)$$
$$\frac{d(-log(L(w)))}{dw_1} = -\sum_{i|y^i = 1}(1 - p(x^i))\cdot x^i + \sum_{i|y^i = 0}p(x^i)\cdot x^i$$

Mit dem Gradientenabstiegsverfahren erhält man für das Beispiel der Kreditvergabe
\begin{center}
  \includegraphics[scale=.25]{ml03_3}
\end{center}

\section{Bayes Klassifikator}
- Der Kredit wird also genau dann ausgegeben, wenn die Wahrscheinlichkeit, dass er zurück gezahlt wird größer ist, als dass er es nicht wird\\
- Der \textit{Bayes Klassifikator} weißt jeder Beobachtung $x \in X$ die wahrscheinlichste Klasse zu: $f(x) =$ $arg$ $max$ $Pr(y = y* | X = x)$\\
- Im Beispiel liegt die \textit{Entscheidungsgrenze} bei 2300EUR
\begin{center}
  \includegraphics[scale=.25]{ml03_4}
\end{center}

\section{Mehrdimensionale Logistische Regression}
- Für $x \in \mathbb{R}^d$ wird das Modell beschrieben durch $p(x) = \frac{e^{w^Tx}}{1 + e^{w^Tx}}$\\
- Der Gradient aus der negativen, logarithmierten Likelihood ergibt sich durch
$$\frac{d(-log(L(w)))}{dw_j} = -\sum_{i|y^i = 1}(1 - p(x_j^i))\cdot x_j^i + \sum_{i|y^i = 0}p(x_j^i)\cdot x_j^i$$

\section{Nichtlineare Logistische Regression}
- Mit der Basiserweiterung $\Phi: \mathbb{R}^d \rightarrow \mathbb{R^{d'}}$ und der mehrdimensionalen
Logistischen Regression können auch nichtlineare Klassifikatoren gelernt werden
\begin{center}
  \includegraphics[scale=.3125]{ml03_5}
\end{center}
- Mit der Basiserweiterung $\Phi(x) = (x_1^2, x_2^2)$ kann ein Logistisches Regressionsmodell gelernt werden, das die beiden Klassen trennt
\begin{center}
  \includegraphics[scale=.3125]{ml03_6}
\end{center}

\section{Leistungsmetriken}
Für die Binäre Klassifikation ist kein $R^2$-Wert möglich $\Rightarrow$ Wahrheitsmatrix:
\begin{center}
  \includegraphics[scale=.25]{ml03_7}
\end{center}
\vspace*{-1.25em}
\begin{itemize}
  \item Genauigkeit (\textit{accuracy}) $\frac{tp + tn}{tp + tn + fp + fn}$: Ist der Anteil der korrekt klassifizierten Daten am Gesamtdatensatz.
  Es wird versucht, die Genauigkeit zu maximieren
  \item Fehlerrate $\frac{fp + fn}{tp + tn + fp + fn}$ = 1 - Genauigkeit: Gegenteil der Genauigkeit. Wird minimiert.
  \item Präzision (\textit{precision}) $\frac{tp}{tp + fp}$: Anteil der korrekt positiv vorhergesagten Datensätze an der Gesamtheit der als positiv vorhergesagten Datensätze
  \item Trefferquote (\textit{recall}) $\frac{tp}{tp + fn}$: Anteil der korrekt positiv vorhergesagten Datensätze an der Gesamtheit der echt positiven Datensätze
\end{itemize}
Präzision und Trefferquote werden maximiert, allerdings muss üblicherweise ein Kompromiss getroffen werden.\\
Es kommt außerdem auf den Anwendungsfall an, welche Leistungsmetrik verwendet werden sollte:\\
\vspace*{-1.25em}
\begin{itemize}
  \item Medizinische Tests: Positiv bedeutet \textit{krank} (z.B. HIV positiv), dann sollte ein Test einen hohen Recall haben
  \item Spam-Erkennung: Positiv bedeutet \textit{gewollte} E-Mail, dann sollte ein Spam-Erkenner eine hohe Precision haben
\end{itemize}

\subsection{Recall-Precision-Kurve}
\begin{center}
  \includegraphics[scale=.225]{ml03_8}
\end{center}

\subsection{Beispiel steigende Kurse an der Börse}
Man setzt auf steigende Kurse immer dann wenn das Modell \textit{up} vorhersagt. Es ist wichtig zu wissen, wie oft das Modell relativ gesehen richtig liegt.
\begin{itemize}
  \item Die Präzision bezieht sich auf den Anteil der korrekt vorhergesagten positiven Renditen. Auf diesen Wert sollte man achten.
  \item Die Genauigkeit besagt in diesem Fall lediglich, welcher Anteil der Vorhersagen richtig war, unabhängig ob \textit{up} oder \textit{down}
  \item Die Trefferquote besagt, welcher Anteil der positiven Renditen tatsächlich auch korrekt vorhergesagt wurde
\end{itemize}

\chapter{Perzeptron und Adaline}
\section{Perzeptron}
\subsection{Biologischer Hintergrund}
In einem biologischen neuronalen Netz finden Berechnungen statt, indem elektrische Ladungen zwischen Nervenzellen ausgetauscht werden\\
\vspace*{-1.25em}
\begin{itemize}
  \item Ein Neuron kann viele Synapsen haben und somit mit hunderten weiteren Neuronen verknüpft sein
  \item Ein Neuron kann auch viele Dendriten besitzen und somit Input von vielen Neuronen empfangen
  \item Verbindungen können \textit{verstärkend} oder \textit{hemmend} wirken, je nach der Chemie innerhalb des synaptischen Spalts
\end{itemize}
\begin{center}
  \includegraphics[scale=.2275]{ml04_3}
\end{center}

\subsection{Einführung}
Ein \textit{Perzeptron} ist ein binärer Klassifikator $f: \mathbb{R}^d \rightarrow\{0,1\}$ definiert als
$$f(x) = \alpha \cdot(w\circ x + w_0)$$
Alternativ wird das Perzeptron in der Literatur auch definiert als
$$f(x) = \alpha \cdot (w\circ x)$$
Hier ist $x = (1, x_1, ..., x_n)^T$ und $w = (w_0, w_1, ..., w_n)^T$, d.h. $x_0 = 1$ und $w_0$ sind in $x$ und $w$ enthalten
\begin{figure}[H]
  \centering
  \begin{minipage}[b]{0.4\textwidth}
    \includegraphics[scale=.2]{ml04_1}
  \end{minipage}
  \hfill
  \begin{minipage}[b]{0.4\textwidth}
    \includegraphics[scale=.325]{ml04_6}
  \end{minipage}
\end{figure}

Die \textit{Heaviside} Aktivierungsfunktion ist definiert als
\begin{equation*}
  \alpha(x) = \begin{cases}
    1 & \text{falls $x > 0$}\\
    0 & \text{andernfalls}
  \end{cases}
\end{equation*}
\begin{center}
  \includegraphics[scale=.225]{ml04_2}
\end{center}

\begin{figure}[H]
  \centering
  \begin{minipage}[b]{0.4\textwidth}
    \includegraphics[scale=.25]{ml04_4}
  \end{minipage}
  \hfill
  \begin{minipage}[b]{0.4\textwidth}
    \includegraphics[scale=.25]{ml04_5}
  \end{minipage}
\end{figure}

Die Klassifizierung lässt sich umdrehen, indem man alle Gewichte mit $\cdot (-1)$ multipliziert

\subsection{Lernalgorithmus}
\begin{itemize}
  \item Beginne mit einer zufälligen oder festen Wahl für $w$, z.B. $w = 0$
  \item Bestimme die falsch klassifzierten Datenpunkte
  \item Versuche iterativ die einzelnen Parameter so zu verändern, dass die Anzahl der falsch klassifzierten Daten sinkt
  \item Höre auf sobald keine Verbesserung mehr eintritt
\end{itemize}

\begin{figure}[H]
  \centering
  \begin{minipage}[b]{0.4\textwidth}
    \includegraphics[scale=.25]{ml04_7}
  \end{minipage}
  \hfill
  \begin{minipage}[b]{0.4\textwidth}
    \includegraphics[scale=.25]{ml04_8}
  \end{minipage}
\end{figure}

\vspace*{-1.25em}
\begin{itemize}
  \item $x$ wurde als Klasse 1 eingestuft, ist aber Klasse 0
  \subitem $f_w(x) = \alpha \cdot (w\circ x) = 1 \Rightarrow w\circ x > 0$
  \subitem $w' = w - x$
  \subitem $w'\circ x = (w - x)\circ x = w\circ x - x \circ x$
  \subitem $f_{w'}(x) = \alpha \cdot (w\circ x - x \circ x) = 0$
  \item $x$ wurde als Klasse 0 eingestuft, ist aber Klasse 1
  \subitem $f_w(x) = \alpha \cdot (w\circ x) = 0 \Rightarrow w\circ x \leq 0$
  \subitem $w' = w + x$
  \subitem $w'\circ x = (w + x)\circ x = w\circ x + x\circ x$
  \subitem $f_{w'}(x) = \alpha(w\circ x + x\circ x) = 1$
\end{itemize}


\begin{lstlisting}
  w = 0
  while (1.0 / n) * sum(|yi - alpha(w.dot(xi))|) > gamma do
    w' = w
    for i = 1 until n
      oi = alpha(w.dot(xi))
      w' += (yi - oi) * xi
    end for
    w = w'
  end while
\end{lstlisting}

Die Gewichte eines Perzeptrons können auch einfacher berechnet werden, wenn z.B. bekannt ist, dass\\
\vspace*{-1.25em}
\begin{itemize}
  \item die Punke $(3, 0)$ und $(0, 3)$ auf der Entscheidungsgrenze (bzw. Entscheidungsoberfläche) liegen
  \item der Ursprung $(0, 0)$ negativ klassifiziert wird
\end{itemize}

$f(x_1, x_2) = w_0 + w_1x_1 + w_2x_2$\\
$f(3, 0) = w_0 + 3x_1 = 0 \Rightarrow w_0 = -3x_1$\\
$f(0, 3) = w_0 + 3x_2 = 0 \Rightarrow w_0 = -3x_2$\\
$f(0, 0) < 0 \Rightarrow x_1 = x_2 = 1 \Rightarrow w_0 = -3$\\
eine Mögliche Lösung ist dann: $f(x_1, x_2) = -3 + x_1 + x_2$

\subsubsection{Beispiel Lernalgorithmus}
\begin{itemize}
  \item $(x^1, y^1) = ([1, 1], 0)$ bzw. $([1, 1, 1], 0)$
  \item $(x^2, y^2) = ([1, 2], 1)$ bzw. $([1, 1, 2], 0)$
  \item $(x^3, y^3) = ([2, 2], 0)$ bzw. $([1, 2, 2], 0)$
\end{itemize}

\begin{tabular}{l|l|l|l|l|l}
  $w_0$ & $w_1$ & $w_2$ & $([1, 1], 0)$ & $([1, 2], 1)$ & $([2, 2], 0)$\\
  \hline
  0 & 0 & 0 & $0^{\checkmark}$ & $0^{x}$ & $0^{\checkmark}$\\
  +1 & +1 & +2 & & &\\
  \hline
  1 & 1 & 2 & $1^x$ & $1^{\checkmark}$ & $1^x$\\
  -1 & -1 & -1 & & &\\
  -1 & -2 & -2 & & &\\
  \hline
  -1 & -2 & -1 & $0^{\checkmark}$ & $0^x$ & $0^{\checkmark}$\\
  +1 & + 1 & + 2 & & &\\
  \hline
  0 & -1 & 1 & $0^{\checkmark}$ & $1^{\checkmark}$ & $0^{\checkmark}$
\end{tabular}

\subsection{Grenzen des Perzeptron}
Probleme wie das Exklusiv-Oder (XOR), welche nicht \textit{linear trennbar} sind, können von einem Perzeptron nicht gelernt werden.\\
Auch wenn ein Problem linear trennbar ist, erhält man mit dem Perzeptron Lernalgorithmus kein eindeutiges Modell
\begin{figure}[H]
  \centering
  \begin{minipage}[b]{0.4\textwidth}
    \includegraphics[scale=.25]{ml04_9}
  \end{minipage}
  \hfill
  \begin{minipage}[b]{0.4\textwidth}
    \includegraphics[scale=.25]{ml04_10}
  \end{minipage}
\end{figure}

\section{Adaline}
Das \textit{Adaline} ähnelt im Aufbau dem Perzeptron besitzt jedoch eine andere Aktivierungsfunktion und einen unterschiedlichen Lernalgorithmus genannt \textit{Deltaregel}
\begin{center}
  \includegraphics[scale=.2125]{ml04_11}
\end{center}
Das \textit{Adaline} ist ein \textit{Binärklassifikator} $f: \mathbb{R}^d \rightarrow {-1, 0, 1}$ definiert als
$$f(x) = \alpha\cdot(w\circ x + w_0)$$
wobei $\alpha$ die \textit{Signum} Aktivierungsfunktion ist.\\
Auch hier nehmen wir implizit an, dass $x_0 = 1$ und $w$ den Parameter $w_0$ beinhaltet

\subsection{Einführung}
Die \textit{Signum} Aktivierungsfunktion ist definiert als
\begin{equation*}
  \alpha(x) = \begin{cases}
    1 & \text{falls $x > 0$}\\
    0 & \text{falls $x = 0$}\\
    -1 & \text{andernfalls}
  \end{cases}
\end{equation*}

\begin{center}
  \includegraphics[scale=.2275]{ml04_12}
\end{center}

\subsection{Lernalgorithmus}
Wie bei der Linearen Regression verwenden wir beim Adaline ein bekanntes Fehlermaß inspiriert
durch die RSS/MSE in Verbindung mit dem \textit{Gradientenabstiegsverfahren}, um den negativen Gradienten
des Fehlermaßes zum Minimum zu folgen:\\
$$E(w)^i = \frac{1}{2}\cdot (y^i - f(x^i))^2 = \frac{1}{2}\cdot(y^i - w\circ x^i)^2$$
$$\frac{dE(w^i)}{dw_j} = (y^i - w\circ x^i)\cdot \frac{d}{dw_j}(y^i - w\circ x^i) = -(y^i - w\circ x^i)$$
$$\nabla_wE(w)^i = \begin{bmatrix}\frac{dE(w)^i}{dw_0}\\...\\\frac{dE(w)^i}{dw_n}\end{bmatrix} = -(y^i - w\circ x^i)\cdot x^i$$

\begin{lstlisting}
  w = 0
  while (1 / n) * sum(|yi - alpha(w.dot(xi))|) > gamma
    for i = 1 until n
      w += eta * (yi - w.dot(xi)) * xi /* Deltaregel */
    end for
  end while
\end{lstlisting}

\begin{itemize}
  \item Bei hoher Lernrate $\eta$ beginnt der Algorithmus zu oszillieren und konvergiert nicht
  \item Aufgrund seiner linearen Natur kann auch das Adaline die XOR-Funktion nicht direkt lernen
  \item Je nach Datensatz und Initialisierung ist der Klassifikator eindeutig
\end{itemize}

\chapter{K-Nearest Neighbors}
\section{Einführung}
Ausgehend vom optimalen Bayes-Klassifikator mit Klassen $C_1, ..., C_m$, $x\in X$
$$f(x) = \text{arg max } Pr(y=C_j|X=x)$$
Bei der KNN-Klassifikation gibt es keine direkte Formel für $Pr(y = C_j|X = x)$.\\
KNN kommt ohne Parameter aus und ist somit eine \textit{nicht-parametrische} Methode.\\
Man merkt sich direkt alle Punkte $x^i$ zusammen mit ihrer Klasse $y^i$, welche im Datensatz
$$T = \{(x^i, y^i)|1\leq i \leq n\}$$
enthalten sind. Es gibt somit keine Trainingsphase. Man nimmt stattdessen an, dass ein zu klassifizierender
Datenpunkt $x$ die gleiche Klasse hat, wie die Trainingsdaten in seiner Nähe

\begin{center}
  \includegraphics[scale=.275]{ml05_1}
\end{center}

Wobei mit \textit{Nähe} der euklidische Abstand

$$dist(x, x') = ||x - x'||_2 = \sqrt{\sum_{i=1}^d(x_i - x'_i)^2}$$

für $x, x' \in \mathbb{R}^d$ gemeint ist.\\
Für eine Menge an Trainingsdaten $T = \{(x^i, y^i) | 1\leq i \leq n\}$ und einen Punkt $x$
definiert man mit $N^k(x)\subseteq T$ die Menge der \textit{k} Trainingspunkte mit der geringsten Distanz $dist(x,x')$ zum Punkt $x$,
die sogenannte \textit{k-Nachbarschaft}. Die Anzahl der Elemente in $N$ ist definiert als $|N^k(x)| = k$.\\
Die Wahrscheinlichkeit der Klassenzugehörigkeit zur Klasse $C_j$ ist definiert als

$$Pr(y = C_j | X = x) = \frac{|\{(x^i, y^i) \in N^k(x) | y^i = C_j\}}{k}$$

\begin{figure}[H]
  \centering
  \begin{minipage}[b]{0.4\textwidth}
    \includegraphics[scale=.275]{ml05_2}
  \end{minipage}
  \hfill
  \begin{minipage}[b]{0.4\textwidth}
    \includegraphics[scale=.275]{ml05_3}
  \end{minipage}
\end{figure}

\begin{center}
  \includegraphics[scale=.275]{ml05_4}
\end{center}

\subsection{Auswirkungen von $K$}
KNN-Methoden besitzen einen einzigen \textit{Hyperparameter} $k \in \mathbb{N}$ welcher maßgeblich die Leistung beeinflusst.
Ziel ist daher, $k$ zu optimieren um\\
\vspace*{-1.5em}
\begin{itemize}
  \item die \textit{Fehlerrate} auf dem Testdatensatz zu minimieren
  \item die Genauigkeit zu maximieren
\end{itemize}

\begin{center}
  \includegraphics[scale=.35]{ml05_5}
\end{center}

Allgemein kann man sagen, dass ein kleines $k$ zu einer rauen Trennkurve zwischen den Klassen führt und größeres $k$ zu einer glatten Trennkurve.
Die Entscheidungsgrenze nähert sich mit steigendem $k$ immer mehr einer Geraden an.\\
Wenn die Entscheidungsgrenze des optimalen Bayes-Klassifikators sehr stark nichtlineare wäre führt demnach ein kleines $k$ zu besseren
Ergebnissen. Zu kleines $k$ birgt jedoch die Gefahr der Überanpassung, zu großes $k$ wiederum kann Unteranpassung zur Folge haben.

\begin{figure}[H]
  \centering
  \begin{minipage}[b]{0.4\textwidth}
    \includegraphics[scale=.265]{ml05_6}
  \end{minipage}
  \hfill
  \begin{minipage}[b]{0.4\textwidth}
    \includegraphics[scale=.255]{ml05_8}
  \end{minipage}
\end{figure}

\subsection{Laufzeit und Beschleunigungsstrukturen}
Ein naiver Ansatz ist bei jedem zu klassifizierenden Datenpunkt die Abstände zu allen $n$ Trainingsdaten zu berechnen,
aufsteigend zu sortieren und die $k$-ersten Punkte auszuwählen. Ein solcher Algorithmus hat die Laufzeitkomplexität von

$$\mathcal{O}(n^2\cdot log(n))$$

Reduzieren der Laufzeit gelingt z.B. mit speziellen Datenstrukturen wie dem \textit{KD-Baum}:

\begin{center}
  \includegraphics[scale=.375]{ml05_7}
\end{center}

Gruppieren der Datenpunkte nach Abstand in einem immer feiner werdenden Netz

\chapter{Support Vector Machines}
\section{Maximal Margin Klassifikator}
\subsection{Einführung}

Maximal Margin Klassifikatoren betrachten die binäre Klassifikation von Punkten $x^i \in \mathbb{R}^d$ mit
$y^i \in \{-1, 1\}$ und nehmen an, dass eine \textit{separierende Hyperebene} im $\mathbb{R}^d$ existiert,
die beide Klassen perfekt voneinander trennt.

\begin{figure}[H]
  \centering
  \begin{minipage}[b]{0.4\textwidth}
    \includegraphics[scale=.275]{ml06_1}
  \end{minipage}
  \hfill
  \begin{minipage}[b]{0.4\textwidth}
    \includegraphics[scale=.275]{ml06_2}
  \end{minipage}
\end{figure}

\subsection{Separierende Hyperebene}
Eine Hyperebene mit Parametern $w_0, w$ welche die Eigenschaft besitzt, dass für alle Datenpunkte
$(x^i, y^i)\in T\subseteq \mathbb{R}^d \times \{-1, 1\}$ gilt, dass
$$w_0 + w^Tx^i > 0 \text{ falls } y^i = 1 \land w_0 + w^Tx^i < 0 \text{ falls } y^i = -1$$
$\Rightarrow y^i\cdot (w_0 + w^Tx^i) > 0$ wird als \textit{separierende Hyperebene} bezeichnet.

\subsection{Maximal Margin}
Mit $f(x) = w_0 + w^Tx$ wird der binäre Klassifikator definiert als $sgn(f(x))$.\\
An $f(x)$ kann man ablesen, wie sicher sich der Klassifikator ist:\\
\vspace*{-1.5em}
\begin{itemize}
  \item Für $|f(x)| \thickapprox 0$ befindet sich der Punkt $x$ sehr nahe an der Hyperebene
  $w_0 + w^Tx = 0$ und ist daher eher ein \textit{Wackelkandidat}
  \item Für $f(x) >> 0$ befindet sich der Punkt (sehr) weit entfernt von der Hyperebene
  im positiven oder negativen Bereich und die Klassifikation ist sicher
\end{itemize}

Die Maximal Margin Hyperebene ist die separierende Hyperebene aus unendlich vielen potentiell separierenden
Hyperebenen, die am weitesten von den Datenpunkten entfernt ist (= Unsicherheit minimeren).

\begin{figure}[H]
  \centering
  \begin{minipage}[b]{0.4\textwidth}
    \includegraphics[scale=.265]{ml06_3}
  \end{minipage}
  \hfill
  \begin{minipage}[b]{0.4\textwidth}
    \includegraphics[scale=.275]{ml06_4}
  \end{minipage}
\end{figure}

Die vier Datenpunkte, die der Maximal Margin Hyperebene am nächsten liegen, nennt man \textit{Support Vektoren},
da sie die Hyperebene stützen. Falls sie verschoben werden, ändert sich auch die Hyperebene. Wichtig ist, dass die
Maximal Margin Hyperebene nur von einer kleinen Anzahl Support Vektoren gestützt wird.

\subsection{Optimierungsproblem}
Um die Maximal Hyperebene zu berechnen, muss das Optimierungsproblem
$$arg\text{ min }\frac{1}{2}\cdot ||w||_2^2$$
unter den Nebenbedingungen
$$y^i\cdot(w_0 + w^Tx^i) \geq 1\text{ }\forall\text{ }1\leq i \leq n$$
gelöst werden. Hier wird der Abstand der Support Vektoren zur Maximal Margin Hyperebene auf $1$ normiert,
d.h. kein Punkt hat einen Abstand $< 1$.\par

\subsection{Karush-Kuhn-Tucker Theoreom}

Das Optimierungsproblem löst man mit dem \textit{Karush-Kuhn-Tucker} Theorem (KKT). Für die \textit{Maxima}
einer Funktion $f(x): \mathbb{R}^d \rightarrow R$ mit $n$ Nebenbedingungen ($1\leq i \leq n$) der Form $g_i(x)\leq 0$ gelten
bezüglich der \textit{Lagrange}-Funktion $L(x) = f(x) - \sum_{i=1}^m\lambda_i\cdot g_i(x)$ die notwendigen Bedingungen:\\
\vspace*{-1.5em}
\begin{itemize}
  \item $\lambda_i\in \mathbb{R}$ und $\lambda_i \geq 0$ (\textit{Lagrange Multiplikatoren})
  \item $\nabla_xL(x) = 0$
  \item $\lambda_i\cdot g_i(x) = 0$ $\forall$ $i\in \{1, ..., n\}$
\end{itemize}

Angewandt auf das Maximal Margin Problem wird nun $-\frac{1}{2}||w||_2^2$ maximiert anstatt $\frac{1}{2}||w||_2^2$ zu minimieren.
Die \textit{Primale} Lagrange-Funktion:
$$L_p(w_0, w) = -\frac{1}{2}||w||_2^2 - \sum_{i=1}^n\lambda_i(1 - y^i\cdot(w_0 + w^Tx^i))$$
Da $-\frac{1}{2}||w||_2^2 = -\frac{1}{2}w^Tw = -\frac{1}{2}(w_1^2 + w_2^2 + ... + w_d^2)$
und $\nabla -\frac{1}{2}||w||_2^2 = -\begin{bmatrix}w_1\\w_2\\...\\w_d\end{bmatrix}$ gilt
$$\nabla_xL_p(w_0, w) = -w + \sum_{i=1}^n\lambda_iy^ix^i = 0$$
wodurch $w = \sum_{i=1}^n\lambda_iy^ix^i$ und $\frac{dL_p(w_0, w)}{dw_0} = \sum_{i=1}^n\lambda_iy^i = 0$.\\
Nach Einsetzen erhält man die \textit{Duale} Lagrange-Funktion:
$$L_d(\lambda) = \sum_{i=1}^n\lambda_i - \frac{1}{2}\sum_{i=1}^n\sum_{j=1}^n\lambda_i\lambda_jy^iy^jx^{i^T}x^j$$
die nun für $\lambda_i \geq 0$ und $i\in \{1,..., n\}$ optimiert wird.

\subsection{Beispiel mit zwei Datenpunkten}
$x^1 = \begin{bmatrix}1\\2\end{bmatrix}, x^2 = \begin{bmatrix}2\\1\end{bmatrix}$ und $y^1 = 1, y^2 = -1$

\begin{center}
  \includegraphics[scale=.275]{ml06_5}
\end{center}

Nach Einsetzen in $L_D(\lambda_1, \lambda_2)$:
$$\lambda_1 + \lambda_2 - \frac{1}{2}\cdot\lambda_1^2\cdot1^2\cdot(1^2 + 2^2) - \frac{1}{2}\cdot\lambda_2^2\cdot(-1)^2\cdot(2^2 + 1^2)
-\frac{1}{2}\cdot\lambda_1\lambda_2\cdot(-1)\cdot(2 + 2) - \frac{1}{2}\cdot\lambda_2\lambda_1\cdot(-1)\cdot1\cdot(2 + 2)$$
$$\Leftrightarrow -\frac{5}{2}\cdot\lambda_1^2 - \frac{5}{2}\cdot\lambda_2^2 + 4\lambda_1\lambda_2 + \lambda_1 + \lambda_2$$
Dies muss nun unter der Nebenbedingung $g(\lambda_1, \lambda_2) = \lambda_1 - \lambda_2 = 0$ optimiert werden. Aus dieser folgt,
dass $\lambda_1 = \lambda_2 = \lambda$ und damit $L_D(\lambda, \lambda) = -\lambda^2 + 2\lambda$ mit globalem Maximum bei $\lambda = 1$

\begin{figure}[H]
  \centering
  \begin{minipage}[b]{0.4\textwidth}
    \includegraphics[scale=.265]{ml06_6}
  \end{minipage}
  \hfill
  \begin{minipage}[b]{0.4\textwidth}
    \includegraphics[scale=.275]{ml06_7}
  \end{minipage}
\end{figure}

$w = \sum_{i=1}^n\lambda_iy^ix^i = \begin{bmatrix}1\\2\end{bmatrix} - \begin{bmatrix}2\\1\end{bmatrix} =
\begin{bmatrix}-1\\1\end{bmatrix}$\\
$w_0$ lässt sich mit den ursprünglichen Nebenbedingungen $y^i\cdot(w_0 + w^Tx^i)\geq 1$\\
$\forall$ $1\leq i\leq n$ bestimmen:\\
\vspace*{-1.5em}
\begin{itemize}
  \item $1\cdot(w_0 + \begin{bmatrix}-1&1\end{bmatrix}\cdot\begin{bmatrix}1\\2\end{bmatrix}) \geq 1$ $\Rightarrow$ $w_0 + 1\geq 1$
  \item $-1\cdot(w_0 + \begin{bmatrix}-1&1\end{bmatrix}\cdot\begin{bmatrix}2\\1\end{bmatrix}) \geq 1$ $\Rightarrow$ $-w_0 + 1\geq 1$
\end{itemize}
Damit ist $w_0 = 0$ und $f(x) = w_0 + w^Tx = x_2 - x_1$

\subsection{Beispiel mit drei Datenpunkten}
$x^1 = \begin{bmatrix}1\\2\end{bmatrix}, x^2 = \begin{bmatrix}2\\4\end{bmatrix}, x^3 = \begin{bmatrix}2\\1\end{bmatrix}$
und $y^1 = 1, y^2 = 1, y^3 = -1$

\begin{center}
  \includegraphics[scale=.275]{ml06_8}
\end{center}

Anschließendes Einsetzen ergibt das Duale Problem der Maximierung von 
$$h(\lambda_1, \lambda_2, \lambda_3) = \lambda_1 + \lambda_2 + \lambda_3 - \frac{5}{2}\lambda_1^2 - 10\lambda_2^2 -
\frac{5}{2}\lambda_3^2 - 10\lambda_1\lambda_2 + 4\lambda_1\lambda_3 + 8\lambda_2\lambda_3$$
mit der Nebenbedingung $g(\lambda_1, \lambda_2, \lambda_3) = \lambda_1 + \lambda_2 - \lambda_3 = 0$.

Nach erneutem Einsetzen erhält man $z(\lambda_1, \lambda_2, \lambda_3, \lambda^*) = h(\lambda_1, \lambda_2, \lambda_3) +
\lambda^*g(\lambda_1, \lambda_2, \lambda_3)$ und das LGS:

$$\nabla_{\lambda_1, \lambda_2, \lambda_3}z(\lambda_1, \lambda_2, \lambda_3, \lambda^*) =
\begin{bmatrix}
  1 & -5\lambda_1 & -10\lambda_2 & 4\lambda_3 & \lambda^*\\
  1 & -20\lambda_2 & -10\lambda_1 & 8\lambda_3 & \lambda^*\\
  1 & -5\lambda_3 & 4\lambda_1 & 8\lambda_2 & -\lambda^*
\end{bmatrix} = 0$$

Nun erhält man $\lambda_1 = \frac{4}{3}, \lambda_2 = -\frac{2}{9}, \lambda_3 = \frac{10}{9}$ und $\lambda^*=-1$.
Da aber $\lambda_2 < 0$ ist dies keine gültige Lösung. Es wird nun an den Grenzen ($\lambda_i = 0$) weitergesucht:

\begin{center}
  \includegraphics[scale=.275]{ml06_9}
\end{center}

Da $g(\lambda_1, \lambda_2, \lambda_3) = \lambda_1 + \lambda_2 - \lambda_3 = 0$, darf $\lambda_3$ nicht $0$ sein, da sonst
$\lambda_1 = \lambda_2 = \lambda_3 = 0$, gültige Lösungen sind daher:\\
\vspace*{-1.5em}
\begin{itemize}
  \item $\lambda_1 = 0 \Rightarrow \lambda_2 = \lambda_3$: $h(\lambda_1, \lambda_2, \lambda_3) = -\frac{9}{2}\lambda_3^2 + 2\lambda_3$
  mit dem Maximum bei $\lambda_2 = \lambda_3 = \frac{2}{9}$
  \item $\lambda_2 = 0 \Rightarrow \lambda_1 = \lambda_3$: $h(\lambda_1, \lambda_2, \lambda_3) = -\lambda_3^2 + 2\lambda_3$
  mit dem Maximum bei $\lambda_1 = \lambda_3 = 1$
\end{itemize}

\begin{center}
  \includegraphics[scale=.275]{ml06_10}
\end{center}

Man erhält mit $\lambda_1 =1, \lambda_2=0, \lambda_3=1$ den selben Klassifikator wie im ersten Beispiel, diesmal ist
der zweite Datenpunkt $x^2$ nicht mehr beteiligt. Man erkennt an den \textit{Lagrange-Multiplikatoren}, welche
Datenpunkte Support Vektoren sind und welche nicht ($\lambda = 0$).

\section{Support Vector Klassifikation}

Die Einschränkung der Maximum Margin Klassifikation, dass Klassen perfekt separierbar sein müssen, wird nun fallen gelassen.
Es ist erlaubt, dass Datenpunkte\\
\vspace*{-1.5em}
\begin{itemize}
  \item innerhalb des Margins
  \item oder sogar auf der falschen Seite der Hyperebene liegen (Grundvoraussetzung für eine trennende Hyperebene)
\end{itemize}

\begin{figure}[H]
  \centering
  \begin{minipage}[b]{0.4\textwidth}
    \includegraphics[scale=.265]{ml06_11}
  \end{minipage}
  \hfill
  \begin{minipage}[b]{0.4\textwidth}
    \includegraphics[scale=.275]{ml06_12}
  \end{minipage}
\end{figure}

\subsection{Optimierungsproblem}

Die Support Vector Klassifizierung liefert das Optimierungsproblem

$$\text{arg max } M$$

unter den Bedingungen

$$\sum_{j=1}^dw_j^2 = 1$$

$$y^i\cdot(w_0 + w^Tx^i)\geq \text{ }M\cdot(1 - \epsilon_i)\text{ },\epsilon\geq 0\text{ }, \forall\text{ }1\geq i\geq n$$

$$\sum_{i=1}^n\epsilon_i\geq C$$

wobei $C\in \mathbb{R}$ und $C \geq 0$ ein \textit{Hyperparameter} ist. Die Schlupfvariable $\epsilon_i$ gibt Auskunft über
die Position des $i$-ten Datenpunkts:\\
\vspace*{-1.5em}
\begin{itemize}
  \item $\epsilon_i = 0$: $x^i$ befindet sich auf der richtigen Seite der Hyperebene
  \item $\epsilon\in\{0,1\}$: $x^i$ verletzt den Mindestabstand zur Hyperebene (Margin), befindet sich jedoch auf der richtigen Seite
  \item $\epsilon_i > 1$: $x^i$ befindet sich auf der falschen Seite der Hyperebene
\end{itemize}

$C$ begrenzt den Grad der \textit{Verletzung}. Ist $C=0$ und damit $\epsilon_1 = ... = \epsilon_n = 0$ handelt es sich um einen
Maximum Margin Klassifikator. Generell dürfen nicht mehr als $C$ Datenpunkte auf der falschen Seite liegen.

\begin{figure}[H]
  \centering
  \begin{minipage}[b]{0.4\textwidth}
    \includegraphics[scale=.265]{ml06_13}
  \end{minipage}
  \hfill
  \begin{minipage}[b]{0.4\textwidth}
    \includegraphics[scale=.275]{ml06_14}
  \end{minipage}
\end{figure}

\begin{figure}[H]
  \centering
  \begin{minipage}[b]{0.4\textwidth}
    \includegraphics[scale=.265]{ml06_15}
  \end{minipage}
  \hfill
  \begin{minipage}[b]{0.4\textwidth}
    \includegraphics[scale=.275]{ml06_16}
  \end{minipage}
\end{figure}

\subsection{Lösen des Optimierungsproblems}
Anwenden des \textit{KKT}, so reduziert sich das Optimierungsproblem auf das duale Problem

$$\text{arg max }L_D(\lambda)$$

mit

$$L_D(\lambda) = \sum_{i=1}^n\lambda_i -\frac{1}{2}\cdot\sum_{i=1}^n\sum_{j=1}^n\lambda_i\lambda_jy^iy^jx^{i^T}x^j$$

mit Nebenbedingung\\
\vspace*{-1.5em}
\begin{itemize}
  \item $\sum_{i=1}^n\lambda_iy^i = $
  \item $0\geq \lambda_i\geq C$ für alle $i\in \{1, ..., n\}$
\end{itemize}

Die Hyperebene wird nur von Datenpunkten innerhalb des Margins bzw. auf der falschen Seite bestimmt.
Sie werden \textit{Support Vektoren} genannt.

\begin{center}
  \includegraphics[scale=.275]{ml06_17}
\end{center}

\section{Support Vector Regression}

Identisch zur \textit{Linearen Regression} geht man bei der \textit{Support Vector Regression} davon aus,
dass die Funktion $f$ die Form $f(x) = w_0 + w^Tx$ mit $w\in \mathbb{R}^d$ und $w_0\in \mathbb{R}$ hat.

\begin{center}
  \includegraphics[scale=.285]{ml06_18}
\end{center}

Zusätzlich fordert man jedoch, dass es einen $\epsilon$-großen Margin um $f$ gibt, in welchem möglichst alle
Datenpunkte liegen. Da dies in den seltesten Fällen klappt, erlaubt man eine \textit{Überschreitung} $\epsilon + \xi_i$ und 
\textit{Unterschreitungen} $\epsilon + \xi_i^*$ des gesamten Marings.\par

Das resultierende quadratische Optimierungsproblem ist für eine feste Wahl von $\epsilon, C\in \mathbb{R} \geq 0$ gegeben durch

$$\text{arg min }\frac{1}{2}||w||^2 + C\cdot \sum_{i=1}^n(\xi_i + \xi_i^*)$$

mit den Nebenbedingungen $\forall\text{ }i\in\{1, ..., n\}$:\\
\vspace*{-1.5em}
\begin{itemize}
  \item $y^i - w^Tx^i - w_0 \leq \epsilon + \xi_i$
  \item $w^Tx^i + w_0 - y^i\leq \epsilon + \xi_i^*$
  \item $\xi_i, \xi_i^* \geq 0$
\end{itemize}

Man bestraft somit nur Abweichungen um $\xi_i$ bzw. $\xi_i^*$ \textit{außerhalb} des $\epsilon$-Margins.
Da nur entweder eine Überschreitung \textit{oder} eine Unterschreitung vorliegt ist immer eins der $\xi_i = 0$.\\
Die \textit{Primale Lagrange Funktion} ist gegeben als:

$$L_P(w, w_0, \lambda, \lambda^*) = -\frac{1}{2}||w||^2 - C\cdot \sum_{i=1}^n(\xi_i + \xi_i^*)
-\sum_{i=1}^n\lambda_i[y^i - w^Tx^i - w_0 - \epsilon - \xi_i]
-\sum_{i=1}\lambda_i^*[w^Tx^i + w_0 - y^i - \epsilon - \xi_i^*]
$$

Nach Anwenden des KKT erhält man:

$$\nabla_wL_P(w, w_0, \epsilon, \epsilon^*) = -w + \sum_{i=1}^n(\lambda_i - \lambda_i^*)\cdot x_i = 0$$

$$\Rightarrow w = \sum_{i=1}^n(\lambda_i - \lambda_i^*)\cdot x^i$$

$$\Rightarrow\frac{dL_P}{dw_0} = \sum_{i=1}^n(\lambda_i - \lambda_i^*) = 0$$

Wenn man diese Erkenntnisse in das ursprüngliche $L_P$ einsetzt, erhält man:

$$L_D(\lambda, \lambda^*) = \frac{1}{2}\sum_{i=1}^n\sum_{j=1}^n(\lambda_i - \lambda_i^*)(\lambda_j - \lambda_j^*)x^{i^T}x^j
+ \epsilon\cdot\sum_{i=1}^n(\lambda_i + \lambda_i^*) + \sum_{i=1}^ny^i(\lambda_i^* - \lambda_i)$$

mit Nebenbedingungen\\
\vspace*{-1.5em}
\begin{itemize}
  \item $\sum_{i=1}^n(\lambda_i - \lambda_i^*) = 0$
  \item $0\leq \lambda_i, \lambda_i^*\leq C$
  \item $\xi_i(C-\lambda_i) = 0$, $\xi_i^*(C - \lambda_i^*) = 0$
\end{itemize}

und zusätzlichen KKT Bedingungen zur Berechnung von $w_0$:\\
\vspace*{-1.5em}
\begin{itemize}
  \item $\lambda_i[y^i - w^Tx^i - w_0 - \epsilon - \xi_i] = 0$
  \item $\lambda_i^*[w^Tx^i + w_0 - y^i - \epsilon - \xi_i^*] = 0$
\end{itemize}

Nur ein Teil der Datenpunkte $x^i \in S\subseteq D$ liefern Lagrange Multiplikatoren $\lambda_i, \lambda_i^* > 0$. Diese
Support Vektoren verletzten den Margin.

\subsection{Kerntrick}

Durch $w = \sum_{i\in S}(\lambda_i - \lambda_i^*)\cdot x^i$ erhält man:

$$f(x) = w^Tx + w_0 = \sum_{i\in S}(\lambda_i - \lambda_i^*)\cdot x^{i^T}x + w_0$$

Das Skalarprodukt $x^{i^T}x$ ist ein Maß für die Ähnlichkeit - umso höher der Wert umso ähnlicher die Vektoren.
Es wird nun zu einem \textit{Kernel} $k$ für eine Basistransformation $\phi$ verallgemeinert:

$$k(u, v) = \phi(u)^T\phi(v)$$

Dadurch erhält man:

$$f(x) = \sum_{i\in S}(\lambda_i - \lambda_i^*)\cdot k(x^i, x) + w_0$$

Die ursprüngliche Formulierung erhält man mit $\phi(u) = u$ und $k(u, v) = u^Tv$.

$$L_d(\lambda, \lambda^*) = \frac{1}{2}\sum_{i=1}^n\sum_{j=1}^n(\lambda_i - \lambda_i^*)(\lambda_j - \lambda_j^*)
k(x^i, x^j) + \epsilon\sum_{i=1}^n(\lambda_i + \lambda_i^*) + \sum_{i=1}^ny^i(\lambda_i^* - \lambda_i)$$

Die Berechnung von $k(u, v)$ ist effizient und kann ohne genaue Kenntniss von $\phi$ erfolgen.
Das Verfahren wird \textit{Kerntrick} genannt. Beispiele:\\
\vspace*{-1.5em}
\begin{itemize}
  \item Linearer Kernel: $k(u, v) = u^Tv$
  \item Polynomialer Kernel: $k(u, v) = (1 + u^Tv)^p$
  \item Radiale Basisfunktion: $k(u, v) = e^{-\frac{||u-v||^2}{2\sigma^2}}$
\end{itemize}

Bei hohen Dimensionen birgt wieder die Gefahr vor Overfitting.

\begin{center}
  \includegraphics[scale=.295]{ml06_19}
\end{center}

Die dicken, schwarzen Punkte sind Support Vektoren (auch diejenigen außerhalb des Margins).

Der Kerntrick lässt sich auch bei der \textit{Support Vector Klassifikation} verwenden:

$$L_D(\lambda) = \sum_{i=1}^n\lambda_i - \frac{1}{2}\sum_{i=1}^n\sum_{j=1}^n\lambda_i\lambda_jy^iy^jk(x^i, x^j)$$

\chapter{Entscheidungsbäume}

\section{Einführung}

Wird angewendet, wenn $X$ \textit{nominal} ist, d.h. mit diskreten Werten aber ohne
natürliche Distanzmetrik  (wie euklidische Norm)), z.B.:\\
\vspace*{-1.5em}
\begin{itemize}
  \item $X_{\text{Farbe}} = \{\text{rot}, \text{grün}, \text{gelb}\}$
  \item $X_{\text{From}} = \{\text{rund}, \text{dünn}\}$
  \item $X_{\text{Größe}} = \{\text{groß}, \text{mittel}, \text{klein}\}$
  \item $X_{\text{Geschmack}} = \{\text{süß}, \text{sauer}\}$
\end{itemize}

Ein Objekt $x\in X$ wird als $d$-Tupel, in diesem Beispiel mit einem $4$-Tupel

$$x = (x_1, x_2, x_3, x_4)\in X$$

mit $X = X_{\text{Farbe}} \times X_{\text{From}} \times X_{\text{Größe}} \times X_{\text{Geschmack}}$\\
Ein Apfel wird z.B. beschrieben durch (rot, rund, mittel, süß)

\subsection{Definition}

Es soll eine Funktion $f: X\rightarrow Y$ gelernt werden, wobei $X$ reell, diskret oder nominal und $Y = \{y_1, ..., y_n\}$
eine diskrete Menge von Klassen ist.\\
Entscheidungsbäume klassifzieren Objekte anhand nominaler Kriterien mit Hilfe von Fragensequenzen, wobei die nächste Frage
direkt von der Antwort der aktuellen Frage abhängt. Die Antwort auf jede Frage muss nominal sein wie z.B. generell $\{$ja, nein$\}$
oder speziell $\{$rot, grün, gelb$\}$.\par
Die Knoten des Baumes sind die systematischen Fragen und die Kanten die jeweiligen Antwortmöglichkeiten. Die Blätter des
Baumes sind die Klassen. Die Klassifikation beginnt in der Wurzel und endet in einem Blatt - also in einer Klasse.\\
Die Kanten, die einen Knoten verlassen müssen \textit{eindeutig} und \textit{erschöpfend} sein, sodass immer genau einer Kante
gefolgt wird.

\begin{center}
  \includegraphics[scale=.285]{ml07_1}
\end{center}

\subsection{Eigenschaften}

\begin{itemize}
  \item Die Klassifikation einzelner Datenpunkte $x\in X$ kann vom Menschen nachvollzogen werden
  \item Die Klassen $y\in Y$ selbst erhalten eine Beschreibung anhand von logischen Kriterien, z.B.
  \subitem Apfel = (Farbe = grün $\land$ Größe = mittel) $\lor$ (Farbe = rot $\land$ Größe = mittel)
  = (Farbe = grün $\lor$ Farbe = rot) $\land$ Größe = mittel
  \item Entscheidungsbäume können daher durch explizites Vorwissen ergänzt werden
\end{itemize}

\subsection{CART}

Das \textit{CART} (Classifikation and Regression Tree) Framework bietet eine allgemeine Methodik umd
verschiedenste Arten von Entscheidungsbäumen zu generieren.\par
Ein Entscheidungsbaum teilt sukzessive die Menge $D \subseteq X \times Y$ in immer kleinere Teilmengen.
Idealerweise endet jeder Pfad in einer \textit{reinen} Menge, d.h. einer Menge $F \subseteq D$ für die gilt,
dass alle Labels $y$ gleich sind (üblicherweise nicht der Fall, ggf. weitere Aufteilung)

\begin{lstlisting}
  D := Trainingsdaten, Q := Fragen
  if stop_criteria(D) then
    return LEAF(compute_class(D))
  else
    foreach q in Q
      Sq = split(D, q)
      delta_iq = compute_improvement(D, S)
    end for
    b = arg max (delta_iq)
    return NODE(b, dtree(S_1b, Q\b), dtree(S_2b, Q\b))
\end{lstlisting}

\section{Aufteilung}

Jede Entscheidung ist mit einem \textit{Split} der Trainingsdaten verbunden. Die Anzahl kann grundsätzlich frei
gewählt werden, bereits zwei Splits reichen allerdings im Allgemeinen aus, d.h. binäre Entscheidungsbäume sind \textit{universell}.

\begin{center}
  \includegraphics[scale=.325]{ml07_2}
\end{center}

Hauptziel ist, einen Baum mit möglichst wenigen Kanten und Knoten zu erstellen. Für jeden Knoten wird daher die Frage gesucht,
die resultierende Datenmengen so \textit{rein} wie möglich macht. Das geschiet über Reduktion der \textit{Unreinheit} (Impurity)
$i(N)$ in Knoten $N$.\\
\vspace*{-1.5em}
\begin{itemize}
  \item $i(N) = 0$: Alle Daten in Knoten $N$ haben die gleiche Klasse
  \item $i(N) = 1$: Alle Klassen sind gleich häufig in Daten in Knoten $N$ vertreten
\end{itemize}

Ein Maß für Unreinheit / Unordnung / Entropie ist

$$i(N) = -\sum_{j=1}^mP(y_j)\cdot log_2(P(y_j))$$

mit\\
\vspace*{-1.5em}
\begin{itemize}
  \item $P(y_j)$ = relative Häufigkeit der Klasse $y_j$ innerhalb der Trainingsdaten an Knoten $N$
  \subitem $P(y_j)$ $\in (0, 1)$, da$log_2(P(y_j)) < 0$ kommt ein $-$ vor der Summe
  \item $m$ = Anzahl Klassen
\end{itemize}

\begin{figure}[H]
  \centering
  \begin{minipage}[b]{0.4\textwidth}
    \includegraphics[scale=.2]{ml07_3}
  \end{minipage}
  \begin{minipage}[b]{0.4\textwidth}
    \includegraphics[scale=.2]{ml07_4}
  \end{minipage}
\end{figure}

\begin{figure}[H]
  \centering
  \begin{minipage}[b]{0.4\textwidth}
    \includegraphics[scale=.2]{ml07_5}
  \end{minipage}
\end{figure}

Weiteres Maß ist die \textit{Gini Unreinheit}:

$$i(N) = \sum_{i\neq j}P(y_i)\cdot P(y_j) = \frac{1}{2}(1 - \sum_{j=1}^mP^2(y_j))$$

und die \textit{Missclassification Impurity} (Wahrscheinlichkeit einen Fehler zu machen):

$$i(N) = 1 - \text{ max }P(y_j)$$

\begin{center}
  \includegraphics[scale=.3]{ml07_6}
\end{center}

An einem Knoten $N$ eines binären Baums möchte man wissen, welche Fragen an die übrigen Testdaten gestellt werden soll.
Eine Heuristik ist die Frage, welche den Rückgang der Unreinheit maximiert:

$$\Delta i(N) = i(N) - P_Pi(N_p) - P_Ni(N_N)$$

mit\\
\vspace*{-1.5em}
\begin{itemize}
  \item $N_P$: Positiver Nachfolgeknoten von $N$
  \item $N_N$: Negativer Nachfolgeknoten von $N$
  \item $P_P = 1 - P_N$: Anteil der Datenpunkte, die dem positiven Knoten zugeordnet werden
  \item Bei nominalen Features muss ein vollständiger Vergleich aller möglichen Fragen pro Knoten in allen
  Dimensionen durchgeführt werden
  \item Bei diskreten / reellen Features werden stattdessen Vergleiche $x_i <= c$, $c\in \mathbb{R}$ verwendet.
  $c$ beschränkt sich auf tatsächlich in den Trainingsdaten vorhandene Werte
\end{itemize}

\section{Beispiel: Erstellung eines binären Entscheidungsbaumes}



\end{document}